\chapter{Objetivos}
\label{ch:objetivos}

Primero enumera los objetivos, no los resumas ni los redactes en un párrafo.  Cada uno de los objetivos de un proyecto debe ser SMART:

\begin{description}
\item[Simple] Cada objetivo tiene que ser independiente, tener sentido por sí mismo y más o menos indivisible.  Si no es suficientemente indivisible, pero tiene sentido como una entidad independiente, debes descomponerlo en subobjetivos.
\item[Medible] Tiene que ser posible medir el grado de consecución al final del TFG.
\item[Acordado] Los objetivos no los pones tú solo.  Deben partir de un acuerdo con tu director.
\item[Realista] No pongas objetivos muy ambiciosos. Basta con que resuelva el problema de la forma más simple posible.  Si superas los objetivos nadie se va a quejar.  El director se encargará de que tampoco sean demasiado poco ambiciosos.
\item[Temporizado] Un objetivo debe tener un marco temporal. Si no es así el objetivo podría no cumplirse nunca.  Es difícil poner límites temporales muy estrictos en un primer proyecto de ingeniería, pero al menos acota.
\end{description}

Tras cada objetivo puedes añadir párrafos ampliando la descripción del objetivo, describiendo los límites y justificándolos.  También puedes describir de qué se parte.  Si es posible debería quedar plenamente justificado que se trata de objetivos SMART.  Considera tanto límites intrínsecos (inherentes a la definición del proyecto) como extrínsecos (limitaciones presupuestarias, equipamiento disponible, etc).